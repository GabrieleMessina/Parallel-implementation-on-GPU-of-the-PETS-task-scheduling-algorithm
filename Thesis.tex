\documentclass[12pt,oneside,a4paper,PisaPhdThesis]{PhdThesis}
\usepackage[centertags]{amsmath}
\usepackage[italian]{babel}
\usepackage{graphics}
\usepackage{graphicx}
\usepackage{color}
\usepackage{hyperref}
\usepackage{epsfig}
\usepackage{amsfonts}
\usepackage{amssymb}
\usepackage{amsthm}
\usepackage{rawfonts}
\usepackage{enumerate}
\usepackage{url}
\usepackage{xspace}
\usepackage{xthesis} 
\usepackage{xtocinc} 

%Necessari per blocchi di codice
\usepackage{listings}    
\usepackage[dvipsnames]{xcolor}      
\definecolor{color1}{HTML}{4B30FF}
\definecolor{dkgreen}{rgb}{0,0.6,0}
\definecolor{gray}{rgb}{0.5,0.5,0.5}
\definecolor{mauve}{rgb}{0.58,0,0.82}

\lstset{frame=tb,
	language=Java,
	aboveskip=3mm,
	belowskip=3mm,
	showstringspaces=false,
	columns=flexible,
	basicstyle={\small\ttfamily},
	numbers=none,
	numberstyle=\tiny\color{gray},
	keywordstyle=\color{blue},
	commentstyle=\color{dkgreen},
	stringstyle=\color{mauve},
	breaklines=true,
	breakatwhitespace=true,
	tabsize=3
}

\lstset { %
	language=C++,
	backgroundcolor=\color{black!5}, % set backgroundcolor
	basicstyle=\footnotesize,% basic font setting
}
%END - Necessari per blocchi di codice

\newlength{\defbaselineskip}
\setlength{\defbaselineskip}{\baselineskip}
\newcommand{\setlinespacing}[1]%
           {\setlength{\baselineskip}{#1 \defbaselineskip}}
\newcommand{\doublespacing}{\setlength{\baselineskip} {2.0 \defbaselineskip}}
\newcommand{\singlespacing}{\setlength{\baselineskip}{\defbaselineskip}}
\newcommand{\mycenterline}[1]{\vspace{.1cm}\newline\vspace{.1cm}\centerline{#1}}


\begin{document}
\selectlanguage{italian}


% Le segueti informazioni devono essere modificate dallo studente
\title {Implementazione parallela su GPU dell'algoritmo di task scheduling PETS}
%\begin{large}Qui \`e Possibile Inserire un Sottotitolo\end{large} %da inserire dentro il tag title
\author{Gabriele Messina}

\date{Anno Accademico 2021/22}


\chaptertitlestyle{serifbig}
\pagestyle{serif}
\maketitle

\begin{frontmatter}

%  inserire di sequito la dedica, altrimenti commentare tutte le righe inserendo un simbolo davanti a ciascuna riga %
%\begin{dedication}
%    in questo punto del testo\\
%    \`e possibile inserire una dedica.
%\end{dedication}

\setlinespacing{1.4}
\pagenumbering{roman}


% questo comando inserisce l'indice dei capitoli
\tableofcontents

\pagenumbering{arabic} \setcounter{page}{1}
% il file introduzione.tex contiene l'abstract in italiano e la sua traduzione nelle due lingue scelte
\chapter*{Abstract}
Qui viene inserita l'introduzione in inglese.
\vspace{50pt}
\hrule
\vspace{50pt}
I sistemi informatici diventano ogni giorno più complessi e le risorse richieste dai processi aumentano di conseguenza, è possibile ottimizzare lo scheduling dei processi in modo da far fronte a questo incremento costante?\\
In queste pagine dimostreremo che è possibile parallelizzare lo scheduling dei processi, sfruttando le capacità computazionali delle GPU, per ottenere un miglioramento nei tempi di scelta del processore ottimale, in particolare questo approccio garantirà ottime prestazioni a sistemi distribuiti in cui il carico di lavoro è elevato o su server che forniscono servizi di cloud computing.
\end{frontmatter}


\setlinespacing{1.4}
% I seguenti file rappresentano i capitoli della tesi.
% Inserire o cancellare le righe per aggiungere o eliminare dei capitoli alla tesi.
\chapter{Titolo del Primo Capitolo}
\vspace{4cm}

\section{Istruzioni generali}
Questo file contiene delle indicazioni di massima su come scrivere una tesi utilizzando latex.
Il file principale \`e denominato \texttt{thesis.tex}. La compilazione dovr\`a avvenire sempre su questo file.

I file \texttt{capitolo1.text}, \texttt{capitolo1.text}, etc., contengono invece i vari capitoli in cui \`e composta la tesi.
Il file \texttt{introduzione.tex} contiene l'abstract della tesi e la sua traduzione nelle due lingue scelte dallo studente. Infine Il file \texttt{bibliografia.tex} contiene la lista dei riferimenti bibliografici.

\section{Elementi di formattazione}
\`E possibile scrivere il testo utilizzando il \emph{carattere corsivo} oppure il \textbf{carattere grassetto}.
In alcuni casi puo essere utile utilizzare il \underline{carattere sottolineato}.

Una lista puntata si realizza nel seguente modo:
\begin{itemize}
	\item primo elemento della lista
	\item secondo elemento della lista
	\item terzo elemento
	\item ultimo elemento della lista
\end{itemize}

Una lista numerata si realizza nel seguente modo:
\begin{enumerate}
	\item primo elemento della lista
	\item secondo elemento della lista
	\item terzo elemento
	\item ultimo elemento della lista
\end{enumerate}

\section{Guide Latex}
\`E possibile trovare dettagliate indicazioni su come scrivere in latex e formattare i documenti all'interno dei file \texttt{ArteLatex.pdf} e \texttt{GuidaGuIT.pdf} che trovate nella directory \texttt{Guide Latex}.
\chapter{Titolo del Secondo Capitolo}
\vspace{4cm}


\section{Note bibliografiche}
\`E possibile inserire delle note bibliografiche\footnote{Questa \`e una nota bibliografica inserita all'interno del testo.} come da esempio.

Le note bibliografiche dovranno essere inserite allo stesso modo\footnote{Simone Faro, Thierry Lecroq: The exact online string matching problem: A review of the most recent results. ACM Comput. Surv. 45(2): 13 (2013)} come mostra il seguente esempio. 
Le note\footnote{Domenico Cantone, Simone Faro, Emanuele Giaquinta: A compact representation of nondeterministic (suffix) automata for the bit-parallel approach. Inf. Comput. 213: 3-12 (2012)} dovranno contenere tutte le informazioni relative ll'articolo citato. 

Tutte le citazioni presenti all'interno delle note del testo dovranno essere riportate all'intrno della sezione bibliografisi.
\section{Ulteriore sezione del capitolo}
Testo del paragrafo.


\section{Come inserire del testo quotato}

Per inserire del testo quotato \`e possibile seguire il seguente esempio

\begin{quote}
\emph{Lorem ipusum}
\end{quote}



\chapter{L'algoritmo PETS}
\vspace{4cm}
L'algoritmo Performance Effective Task Scheduling (PETS), presentato in [], è un'euristica che si propone di ottenere risultati migliori rispetto alle controparti già presenti in letteratura, vedi HEFT e CPOP, in termini di costo computazionale e ottimizzazione generale delle risorse.
\\
L'algoritmo è diviso in tre fasi:
\begin{itemize}
	\item Ordinamento sulla base del livello
	\item Calcolo delle priorità
	\item Selezione del processore 
\end{itemize}

\newpage

\section{Ordinamento sulla base del livello}
In questa fase si attraversa la DAG attraverso una ricerca in ampiezza (BFS), partendo quindi dal task con tempo di inizio minore (o entry-task), con l'obiettivo di raggruppare nello stesso livello task che fra loro sono indipendenti e che quindi possono essere eseguiti in parallelo.
In particolare, per ogni task, il livello ad esso associato è dato dal livello massimo dei suoi predecessori (nodi della DAG da cui il task dipende) incrementato di uno. I task che non hanno predecessori sono detti entry-task o entrypoint e sono assegnati al livello 0.


\section{Calcolo della priorità}
Per ogni task si procede quindi a calcolare la sua priorità o il suo grado in modo da determinare all'interno di ogni livello l'ordine ottimale di esecuzione dei task, l'ordine cioè che fornisce uno scheduling finale migliore.
Per fare questo si definiscono tre proprietà:
\begin{itemize}
	\item Average Computation Cost (ACC): costo medio di computazione del task sui processori.
	\item Data Transfer Cost (DTC): quantità di dati da trasferire dal task a tutti i suoi successori.
	\item Rank of Predecessor Task (RPT): il grado più alto fra i predecessori del task.
\end{itemize}

Il grado è calcolato attraverso la seguente equazione:
\begin{equation}\label{Grado}
	rank(v\ped{i}) = round(ACC(v\ped{i}) + DTC(v\ped{i}) + RPT(v\ped{i}))
\end{equation}

Quindi, per ogni livello, i task con grado più alto ricevono priorità maggiore.



\section{Selezione del processore}
A questo punto si scandiscono i task sulla base della loro priorità e, ogni task, viene associato al processore migliore, cioè quello che fornisce $EFT$ minore. 
Dove $EFT$ è definita come segue:
\begin{displaymath}
	avail[j] := \text{prossimo istante in cui il processore $j$ è libero}
\end{displaymath}
\begin{displaymath}
	C\ped{t,i} := \text{tempo necessario al trasferimento dei dati tra i task $t$ e $i$} 
\end{displaymath}
\begin{displaymath}
	W\ped{i,j} := \text{costo computazionale del processo $i$ sul processore $j$} 
\end{displaymath}
\begin{equation}\label{EST}
	EST(v\ped{i},p\ped{j}) = max(avail[j], max(AFT(v\ped{t}) + C\ped{t,i} : V\ped{t} \in pred(v\ped{i})))
\end{equation}
\begin{equation}\label{EFT}
	EFT(v\ped{i},p\ped{j}) = W\ped{i,j} + EST(v\ped{i},p\ped{j})
\end{equation}


\section{Complessità computazionale}
La prima fase dell'algoritmo ha complessità $O(e)$ se si implementa la DAG attraverso liste di adiacenza.
\\
L'ordinamento dei task ha complessità $O(\log(v))$ se si implementa una coda di priorità attraverso un heap binario.
\\
Infine per ogni task in coda si calcola il costo su ogni processore, quindi la complessità computazionale dell'algoritmo PETS è \( O(e)(p + \log{v}) \). \textcolor{red}{\( O(e)(p * \log{v}) \)?}
\\
Con $v$ numero di task, $e$ numero di archi e $p$ numero di processori.
\chapter{L'implementazione in OpenCL}
\vspace{4cm}


\section{Generatore dei task}
Al fine di poter comparare i risultati ottenuti nel paper[jcssp] con i nostri, si è deciso di implementare un generatore di task con caratteristiche simili a quelle del paper, che fornisca in output una varietà di DAG su cui poter eseguire i test.
\\
Il generatore dipende da diversi parametri di input che sono:
\begin{itemize}
	\item{Number of tasks in the graph ($v$)}
	\item{Out degree ($\beta$)}
	\item{Shape parameter of a graph ($\alpha$)}
	\item{Communication to Computation Ratio (CCR)}
	\item{Range percentage of computation cost ($\eta$)}
\end{itemize}

In particolare, l'altezza del grafo è generata randomicamente a partire da una distribuzione uniforme con valore medio pari a $\frac{\sqrt{v}}{\alpha}$, mentre l'ampiezza a partire da una distribuzione uniforme con valore medio pari a $\sqrt{v}\times\alpha$.
In questo modo si può generare un grafo più o meno denso semplicemente modificando il parametro $\alpha$.

%TODO: Aggiungere descrizione parametro beta

Inoltre il parametro CCR indica quanto i task sono impattanti dal punto di vista computazionale rispetto alla mole di dati che analizzano, se CCR è molto basso il costo computazionale è più alto rispetto alla quantità di dati trasmessi ai task successori pertanto l'applicazione può essere considerata ad alta intensità di calcolo, viceversa se CCR è alto l'applicazione trasmette molti dati tra i task ma non è molto pesante dal punto di vista computazionale.

Infine il parametro $\eta$ indica il grado di eterogeneità del sistema, cioè se ci sono differenze significative tra le prestazioni dei processori, se $\eta$ è alto i costi dei task sui processori variano molto di processore in processore, viceversa se $\eta$ basso tutti i processori completeranno lo stesso task in tempi uguali. 
A partire da questo, il costo medio della computazione $W\ped{i}$ per ogni task è stato scelto casualmente da una distribuzione uniforme con estremi $0$ e $2 \times \text{Wdag}$, dove Wdag è una costante del generatore che indica il costo medio di computazione dei task del grafo, e il costo computazionale $W\ped{i,j}$ di ogni task $v\ped{i}$ su ogni processore $p\ped{j}$ è scelto randomicamente dall'intervallo $[W\ped{i}\times (1-\frac{\eta}{2}), W\ped{i}\times (1+\frac{\eta}{2})]$.

\section{Fasi dell'algoritmo}
Al fine di implementare attraverso OpenCL una versione parallelizzata dell'algoritmo PETS, si è suddiviso l'algoritmo in 4 fasi:

\begin{enumerate}
	\item Ricerca degli entrypoint
	\item Calcolo dei livelli e dei ranghi
	\item Ordinamento dei task
	\item Selezione del processore
\end{enumerate}


Per ogni fase è stato implementato un kernel OpenCL da eseguire in parallelo su GPU.

\subsection{Ricerca degli entrypoint}
\begin{lstlisting}[language=C++, caption={Find entrypoints kernel II},captionpos=b]
	kernel void entry_discover_rectangular(const int n_nodes, global edge_t* restrict edges, volatile global int* n_entries, global int* entries)
	{
		int current_node_index = get_global_id(0);
		if (current_node_index >= n_nodes) return;
		
		if (edges[matrixToArrayIndex] <= -1)
		entries[i] = 1;
	}
\end{lstlisting}

\subsection{Calcolo dei livelli e dei ranghi}
\begin{lstlisting}[language=C++, caption={Compute metrics kernel II},captionpos=b]
	kernel void compute_metrics_rectangular(global int* restrict nodes, global int* queue_, global int* next_queue_, const int n_nodes, global edge_t* restrict edges, global edge_t* restrict edges_reverse, volatile global int2* metriche, const int max_adj_dept)
	{
		int current_node_index = get_global_id(0);
		if(current_node_index >= n_nodes) return;
		
		[...] //omissis of various security checks
		
		for (int j = 0; j < max_adj_dept; j++) {
			int parentAdjIndex = j;
			matrixToArrayIndex = matrix_to_array_indexes(parentAdjIndex, current_node_index, n_nodes);
			int edge_weight = 1;
			int parent_index = edges[matrixToArrayIndex];
			if (parent_index >= 0){
				int weight_with_this_parent = edge_weight + metriche[parent_index].x + nodes[current_node_index];
				int level_with_this_parent = metriche[parent_index].y + 1;
				metrics_with_this_parent = (int2)(weight_with_this_parent, level_with_this_parent);
				if (gt(metrics_with_this_parent, metriche[current_node_index]))
				metriche[current_node_index] = metrics_with_this_parent;
			}
			int child_index = edges_reverse[matrixToArrayIndex];
			if (child_index >= 0)
			atomic_inc(&next_queue_[child_index]);
		}
	}
\end{lstlisting}

\subsection{Ordinamento dei task}
\begin{lstlisting}[language=C++, caption={MergeSort kernel for metrics couple array, source: \url{https://github.com/Gram21/GPUSorting}},captionpos=b]
	__kernel void merge_sort(const __global int2* inArray, __global int2* outArray, const uint stride, const uint size)
	{
		const uint baseIndex = get_global_id(0) * stride;
		if ((baseIndex + stride) > size) return;
		const char dir = 1;
		uint middle = baseIndex + (stride >> 1);
		uint left = baseIndex;
		uint right = middle;
		bool selectLeft;
		
		for (uint i = baseIndex; i < (baseIndex + stride); i++) {
			selectLeft = (left < middle && (right == (baseIndex + stride) || lte(inArray[left], inArray[right]))) == dir;
			
			outArray[i] = (selectLeft) ? inArray[left] : inArray[right];
			
			left += selectLeft;
			right += 1 - selectLeft;
		}
	}
\end{lstlisting}

\subsection{Selezione del processore}
\begin{lstlisting}[language=C++, caption={Compute metrics kernel II},captionpos=b]
	void ScheduleTasksOnProcessors()
	{
		for (int i = 0; i < metrics_len; i++)
		{
			int current_node = ordered_metrics[i].z; 
			if (current_node >= n_nodes) continue;
			int predecessor_with_max_aft = -1;
			int max_aft_of_predecessors = -1;
			int processor_for_max_aft_predecessor = -1;
			int weight_for_max_aft_predecessor = 0;

			for (int j = 0; j < DAG->max_parents_for_nodes; j++)
			{
				int currentParent = edges[matrix_to_array_indexes(j, current_node, DAG->len)];
				if (currentParent > -1) {
					int edge_weight_with_parent = predecessors[matrix_to_array_indexes(j, current_node, DAG->len)];
					int parentEFT = task_processor_assignment[currentParent].z + edge_weight_with_parent;
					if (parentEFT > max_aft_of_predecessors) {
						max_aft_of_predecessors = parentEFT;
						predecessor_with_max_aft = currentParent;
						processor_for_max_aft_predecessor = task_processor_assignment[currentParent].x;
						weight_for_max_aft_predecessor = edge_weight_with_parent;
					}
				}
			}
			
			int eft_min = INT_MAX;
			
			int cost_of_predecessors_in_different_processors = 0;
			int remaining_transfer_cost = 0;
			for (int j = 0; j < DAG->max_parents_for_nodes; j++)
			{
				int currentParent = edges[matrix_to_array_indexes(j, current_node, DAG->len)];
				if (currentParent > -1 && currentParent != predecessor_with_max_aft) {
					cost_of_predecessors_in_different_processors = max(
					cost_of_predecessors_in_different_processors,
					task_processor_assignment[currentParent].z + predecessors[matrix_to_array_indexes(j, current_node, DAG->len)]);
				}
			}
			
			for (int processor = 0; processor < DAG->number_of_processors; processor++) {
				int cost_of_predecessor_in_same_processor = 0;
				int cost_on_processor = costs[matrix_to_array_indexes(current_node, processor, DAG->number_of_processors)];
				if (processor_for_max_aft_predecessor == processor) {
					cost_of_predecessor_in_same_processor = weight_for_max_aft_predecessor;
				}
				remaining_transfer_cost = max(max_aft_of_predecessors - cost_of_predecessor_in_same_processor, cost_of_predecessors_in_different_processors);
				
				int est = max(processorsNextSlotStart[processor], remaining_transfer_cost);
				int eft = est + cost_on_processor;
				if (eft < eft_min) {
					eft_min = eft;
					task_processor_assignment[current_node] = cl_int3{ processor, est, eft };
				}
			}
			
			processorsNextSlotStart[task_processor_assignment[current_node].x] = task_processor_assignment[current_node].z;
		}
	}
\end{lstlisting}
\section{Ottimizzazioni}
\chapter{Risultati}
\vspace{4cm}



\begin{frontmatter}
\pagenumbering{arabic} 
\setcounter{page}{52}
\chapter{Conclusioni}

Inserire qui le conclusioni alla tesi.
\include{bibliografia}
\end{frontmatter}
\end{document}
