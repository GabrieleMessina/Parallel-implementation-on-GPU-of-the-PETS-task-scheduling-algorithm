\chapter{Titolo del Primo Capitolo}
\vspace{4cm}

\section{Istruzioni generali}
Questo file contiene delle indicazioni di massima su come scrivere una tesi utilizzando latex.
Il file principale \`e denominato \texttt{thesis.tex}. La compilazione dovr\`a avvenire sempre su questo file.

I file \texttt{capitolo1.text}, \texttt{capitolo1.text}, etc., contengono invece i vari capitoli in cui \`e composta la tesi.
Il file \texttt{introduzione.tex} contiene l'abstract della tesi e la sua traduzione nelle due lingue scelte dallo studente. Infine Il file \texttt{bibliografia.tex} contiene la lista dei riferimenti bibliografici.

\section{Elementi di formattazione}
\`E possibile scrivere il testo utilizzando il \emph{carattere corsivo} oppure il \textbf{carattere grassetto}.
In alcuni casi puo essere utile utilizzare il \underline{carattere sottolineato}.

Una lista puntata si realizza nel seguente modo:
\begin{itemize}
	\item primo elemento della lista
	\item secondo elemento della lista
	\item terzo elemento
	\item ultimo elemento della lista
\end{itemize}

Una lista numerata si realizza nel seguente modo:
\begin{enumerate}
	\item primo elemento della lista
	\item secondo elemento della lista
	\item terzo elemento
	\item ultimo elemento della lista
\end{enumerate}

\section{Guide Latex}
\`E possibile trovare dettagliate indicazioni su come scrivere in latex e formattare i documenti all'interno dei file \texttt{ArteLatex.pdf} e \texttt{GuidaGuIT.pdf} che trovate nella directory \texttt{Guide Latex}.