\chapter{Titolo del Primo Capitolo}
\vspace{4cm}

\section{Istruzioni generali}
Questo file contiene delle indicazioni di massima su come scrivere una tesi utilizzando latex.
Il file principale \`e denominato \texttt{thesis.tex}. La compilazione dovr\`a avvenire sempre su questo file.

I file \texttt{capitolo1.text}, \texttt{capitolo1.text}, etc., contengono invece i vari capitoli in cui \`e composta la tesi.
Il file \texttt{introduzione.tex} contiene l'abstract della tesi e la sua traduzione nelle due lingue scelte dallo studente. Infine Il file \texttt{bibliografia.tex} contiene la lista dei riferimenti bibliografici.

\section{Elementi di formattazione}
\`E possibile scrivere il testo utilizzando il \emph{carattere corsivo} oppure il \textbf{carattere grassetto}.
In alcuni casi puo essere utile utilizzare il \underline{carattere sottolineato}.

Una lista puntata si realizza nel seguente modo:
\begin{itemize}
	\item primo elemento della lista
	\item secondo elemento della lista
	\item terzo elemento
	\item ultimo elemento della lista
\end{itemize}

Una lista numerata si realizza nel seguente modo:
\begin{enumerate}
	\item primo elemento della lista
	\item secondo elemento della lista
	\item terzo elemento
	\item ultimo elemento della lista
\end{enumerate}

\section{Guide Latex}
\`E possibile trovare dettagliate indicazioni su come scrivere in latex e formattare i documenti all'interno dei file \texttt{ArteLatex.pdf} e \texttt{GuidaGuIT.pdf} che trovate nella directory \texttt{Guide Latex}.




\chapter{Titolo del Secondo Capitolo}
\vspace{4cm}


\section{Note bibliografiche}
\`E possibile inserire delle note bibliografiche\footnote{Questa \`e una nota bibliografica inserita all'interno del testo.} come da esempio.

Le note bibliografiche dovranno essere inserite allo stesso modo\footnote{Simone Faro, Thierry Lecroq: The exact online string matching problem: A review of the most recent results. ACM Comput. Surv. 45(2): 13 (2013)} come mostra il seguente esempio. 
Le note\footnote{Domenico Cantone, Simone Faro, Emanuele Giaquinta: A compact representation of nondeterministic (suffix) automata for the bit-parallel approach. Inf. Comput. 213: 3-12 (2012)} dovranno contenere tutte le informazioni relative ll'articolo citato. 

Tutte le citazioni presenti all'interno delle note del testo dovranno essere riportate all'intrno della sezione bibliografisi.
\section{Ulteriore sezione del capitolo}
Testo del paragrafo.


\section{Come inserire del testo quotato}

Per inserire del testo quotato \`e possibile seguire il seguente esempio

\begin{quote}
	\emph{Lorem ipusum}
\end{quote}


\chapter{Titolo del Terzo Capitolo}
\vspace{4cm}


\section{Prima sezione}
Testo del paragrafo.

\section{Seconda sezione}
Testo del paragrafo.

\section{Terza sezione}
Testo del paragrafo.





