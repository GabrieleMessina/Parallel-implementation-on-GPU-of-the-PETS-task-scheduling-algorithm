\chapter{L'algoritmo PETS}
\vspace{4cm}
L'algoritmo Performance Effective Task Scheduling (PETS), presentato in \citetitle{ilavarasan2007low}\cite{ilavarasan2007low}, è un'euristica che si propone di ottenere risultati migliori rispetto alle controparti già presenti in letteratura (vedi HEFT\cite{993206} e CPOP\cite{993206}) in termini di costo computazionale e ottimizzazione generale delle risorse.
\\
L'algoritmo è diviso in tre fasi:
\begin{itemize}
	\item Ordinamento sulla base del livello
	\item Calcolo delle priorità
	\item Selezione del processore 
\end{itemize}

\newpage

\section{Ordinamento sulla base del livello}
In questa fase si attraversa la DAG attraverso una ricerca in ampiezza (BFS), partendo quindi dal task con tempo di inizio minore (o entry-task), con l'obiettivo di raggruppare nello stesso livello task che fra loro sono indipendenti e che quindi possono essere eseguiti in parallelo.
In particolare, per ogni task il livello ad esso associato è dato dal livello massimo dei suoi predecessori (nodi della DAG da cui il task dipende) incrementato di uno. I task che non hanno predecessori sono detti entry-task o entrypoint e sono assegnati al livello 0.


\section{Calcolo della priorità}
Per ogni task si procede quindi a calcolare la sua priorità o il suo grado in modo da determinare all'interno di ogni livello l'ordine ottimale di esecuzione dei task, l'ordine cioè che fornisce uno scheduling finale migliore.
Per fare questo si definiscono tre proprietà:
\begin{itemize}
	\item Average Computation Cost (ACC): costo medio di computazione del task sui processori.
	\item Data Transfer Cost (DTC): quantità di dati da trasferire dal task a tutti i suoi successori.
	\item Rank of Predecessor Task (RPT): il grado più alto fra i predecessori del task.
\end{itemize}
Il grado è calcolato attraverso la seguente equazione:
\begin{equation}\label{Grado}
	rank(v\ped{i}) = round(ACC(v\ped{i}) + DTC(v\ped{i}) + RPT(v\ped{i}))
\end{equation}
Quindi, per ogni livello, i task con grado più alto ricevono priorità maggiore.



\section{Selezione del processore}
A questo punto si scandiscono i task sulla base della loro priorità e, ogni task, viene associato al processore migliore, cioè quello che fornisce $EFT$ (Earliest Finish Time) minore. 
Dove $EFT$ è definito come segue:
\begin{displaymath}
	avail[j] := \text{prossimo istante in cui il processore $j$ è libero.}
\end{displaymath}
\begin{displaymath}
	C\ped{t,i} := \text{tempo necessario al trasferimento dei dati tra i task $t$ e $i$.} 
\end{displaymath}
\begin{displaymath}
	W\ped{i,j} := \text{costo computazionale del processo $i$ sul processore $j$.} 
\end{displaymath}
\begin{equation}\label{EST}
	EST(v\ped{i},p\ped{j}) = max(avail[j], max(EFT(v\ped{t}) + C\ped{t,i} : v\ped{t} \in pred(v\ped{i})))
\end{equation}
\begin{equation}\label{EFT}
	EFT(v\ped{i},p\ped{j}) = W\ped{i,j} + EST(v\ped{i},p\ped{j})
\end{equation}


\section{Complessità computazionale}
La visita in ampiezza ha complessità $O(v + e)$ se si implementa la DAG attraverso liste di adiacenza.
\\
L'ordinamento dei task ha complessità $O(\log(v))$ se si implementa una coda di priorità attraverso un heap binario.
\\
Infine per ogni task in coda si calcola il costo su ogni processore, quindi la complessità computazionale dell'algoritmo PETS è \( O(e)(p + \log{v}) \).
%TODO: controllare questa analisi
\\
Con $v$ numero di task, $e$ numero di archi e $p$ numero di processori.